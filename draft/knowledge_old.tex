These motivational underpinnings, in turn, have been a subject of lively research in psychology and adjacent fields. In her influential article, \citet{kunda1990case} distinguishes between reasoning driven primarily by directional as compared to accuracy goals; people who engage in accuracy-driven reasoning process information more carefully and tend to rely less on biased strategies or cognitive shortcuts. Importantly, experimental evidence suggests that accuracy goals can be activated by telling participants that they have to \textit{justify their beliefs} in front of others \citep{kunda1990case}. There are several notable examples in political science where researchers rely on this strategy to induce in-depth processing by creating the expectation among participants that they have to explain their decisions at some point in the study \citep[e.g.,][]{tetlock1983accountability,redlawsk2002hot,eveland2004effect,bolsen2014influence}.
% also see tetlock1989social

%%%%%%%%%%%%%%%%%%%%%%

Rather than trying to develop a new item battery that addresses some of these issues, I propose an alternative approach. Instead of testing respondents on a specific set of predetermined facts, we can make inferences about political sophistication by analyzing how they discuss their attitudes and beliefs in their own words. 

%I argue that we can turn this logic on its head in order to assess the level of elaboration in political reasoning in public opinion surveys. 
I argue that we can extend this logic to assess the degree to which people engage in elaborate reasoning about a political issue by examining \textit{how} they talk about and justify their preferences \citep[see also][]{rosenberg1988structure,rosenberg1988political}. For example, if respondents are motivated and able to engage in in-depth processing to form quality opinions on a specific topic, they should approach it from multiple perspectives and show awareness of arguments for and against certain positions \citep{cappella2002argument}.\footnote{A similar argument is made by \citet{colombo2016justifications} who investigates the competence of Swiss citizens voting in policy referenda. Colombo conceptualizes competence as a voter's ability to justify his or her political decisions, and measures the concept by manually coding open-ended responses to survey questions.} In other words, how people talk about their political preferences provides insights into their underlying motivation to engage in in-depth reasoning and may ultimately allow us to make inferences about their level of sophistication in a specific issue domain.
%Rather than using people's expectation to provide justifications as a manipulation for accuracy motivations, we can therefore directly assess the level of elaboration in political reasoning by examining how people discuss their own views.

Rather than having to devise a new set of questions that attempt to capture information necessary to make competent decisions, we can simply analyze how respondents describe and justify their political preferences in verbatim.

%There is an additional reason why it is important to consider how people talk about their political preferences when examining citizen competence and sophistication. Political information often reaches citizens indirectly through conversations with coworkers, friends, and family \citep[see][for a recent example]{druckman2018no}. Studies have further shown that political knowledge itself is transmitted through social interactions, especially since people are able to seek out politically competent discussants \citep{huckfeldt2001social,eveland2009political}. Of course, this information diffusion is fundamentally grounded in how people discuss and justify their political beliefs when talking to each other. A survey---especially if it is conducted face to face---could in theory be characterized as such a \textit{formalized} conversation between two individuals (\citealt{sudman1996thinking}; see also \citealt{grice1975logic,grice1978further}). How people discuss their beliefs in open-ended responses can therefore be informative for individual political competence as well as its crucial role in knowledge transmission through social interaction. 