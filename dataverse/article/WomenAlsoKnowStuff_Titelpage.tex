\maketitle
\thispagestyle{empty}

%\begin{center}
%-- WORK IN PROGRESS -- \\
%PLEASE DO NOT CITE OR REDISTRIBUTE WITHOUT PERMISSION
%\end{center} 

\hfill
\begin{abstract}\singlespacing
	\noindent 
	This article proposes a simple but powerful framework to measure political sophistication based on open-ended survey responses. \textit{Discursive sophistication} uses automated text analysis methods to capture the complexity of individual attitude expression. I validate the approach by comparing it to conventional political knowledge metrics using different batteries of open-ended items across five surveys spanning four languages (total $N \approx 35,000$). The new measure casts doubt on the oft-cited gender gap in political knowledge: Women might know fewer facts about institutions and elites, but they do not differ substantively in the sophistication of their expressed political attitudes.
	
	\vspace{\baselineskip}
	\noindent \textit{Keywords}: political sophistication, gender gap, open-ended responses, text analysis
	
	\vspace{\baselineskip}
	%\noindent \textit{Word count}: 5873 (via TeXcount)
\end{abstract}
\hfill