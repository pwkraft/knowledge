\section*{Attitude Expression and Elaborate Reasoning}
% Style over Substance?
% In-Depth Processing and Attitude Expression

% OUTLINE THEORY %
% 1. Everyone uses knowledge, and it's usually off-the-shelf stuff
% 2. Since it's infeasible to come up with new knowledge battery for every problem, Druckman argues that we should focus on the process rather than the content/substance
% 3. Previous researchers have induced in-depth processing by asking people to justify their preferences
% 4. I ask the question of how people should justify their preferences if they engage in in-depth processing! 
% 5. Examining how people justify their preferences is also important because that's how social infleunce etc. is transmitted.
% 6. (optional) Segway to discuss how these attributes of elaborate processing and sophisticated belief systems should be expressed in open-ended responses.

%In modern democracies, citizens can engage in politics through various means such as voting in local, state, or federal elections. Depending on the institutional setup, they may also directly decide on specific policies through referenda. A common concern in these contexts is whether citizens are able to make high quality decisions in accordance with their underlying interests. Given that

%Rather than trying to develop recall items that presupposes a set of facts as necessary for political competence, I therefore analyze \textit{how} individuals discuss their preferences related to a given political task.

Most studies on political attitudes and public opinion consider individual political knowledge in one way or another---either directly as an outcome variable of interest, as a major explanatory factor, or as an important confounder to control for. In order to measure the underlying latent concept, researchers commonly rely on short batteries of standard recall questions on basic facts about the political system.

To be fair, it is not always feasible for researchers to develop new sets of knowledge items that specifically target relevant information to make competent decisions in any particular context. Given that there is usually no consensus about what information is necessary in the first place, \citet{druckman2014pathologies} proposes abandoning recall questions as measures of ``quality opinion.'' Instead, the author advocates ``\textit{less} focus on the \textit{content/substance} of opinions [...] and \textit{more} on the \textit{process} and specifically the \textit{motivation} that underlies the formation of those opinions'' \citeyearpar[478, emphasis in the original]{druckman2014pathologies}. The key distinction should therefore be how citizens approach a political issue and whether they are motivated to engage in elaborate reasoning to arrive at their particular decision.
% a useful alternative is to concentrate on whether people are motivated to engage in elaborate reasoning when forming their preferences.

These motivational underpinnings, in turn, have been a subject of lively research in psychology and adjacent fields. In her influential article, \citet{kunda1990case} distinguished between reasoning driven primarily by directional as compared to accuracy goals; people who engage in accuracy-driven reasoning process information more carefully and tend to rely less on biased strategies or cognitive shortcuts. Importantly, experimental evidence suggests that accuracy goals can be activated by telling participants that they have to \textit{justify their beliefs} in front of others \citep{kunda1990case}. There are several notable examples in political science where researchers rely on this strategy to induce in-depth processing by creating the expectation among participants that they have to explain their decisions at some point in the study \citep[e.g.,][]{tetlock1983accountability,redlawsk2002hot,eveland2004effect,bolsen2014influence}.
% also see tetlock1989social

%I argue that we can turn this logic on its head in order to assess the level of elaboration in political reasoning in public opinion surveys. 
I argue that we can extend this logic to assess the degree to which people engage in elaborate reasoning about a political issue by examining \textit{how} they talk about and justify their preferences \citep[see also][]{rosenberg1988structure,rosenberg1988political}. For example, if respondents are motivated and able to engage in in-depth processing to form quality opinions on a specific topic, they should approach it from multiple perspectives and show awareness of arguments for and against certain positions \citep{cappella2002argument}.\footnote{A similar argument is made by \citet{colombo2016justifications} who investigates the competence of Swiss citizens voting in policy referenda. Colombo conceptualizes competence as a voter's ability to justify his or her political decisions, and measures the concept by manually coding open-ended responses to survey questions.} In other words, how people talk about their political preferences provides insights into their underlying motivation to engage in in-depth reasoning and may ultimately allow us to make inferences about their level of sophistication in a specific issue domain.
%Rather than using people's expectation to provide justifications as a manipulation for accuracy motivations, we can therefore directly assess the level of elaboration in political reasoning by examining how people discuss their own views.

%There is an additional reason why it is important to consider how people talk about their political preferences when examining citizen competence and sophistication. Political information often reaches citizens indirectly through conversations with coworkers, friends, and family \citep[see][for a recent example]{druckman2018no}. Studies have further shown that political knowledge itself is transmitted through social interactions, especially since people are able to seek out politically competent discussants \citep{huckfeldt2001social,eveland2009political}. Of course, this information diffusion is fundamentally grounded in how people discuss and justify their political beliefs when talking to each other. A survey---especially if it is conducted face to face---could in theory be characterized as such a \textit{formalized} conversation between two individuals (\citealt{sudman1996thinking}; see also \citealt{grice1975logic,grice1978further}). How people discuss their beliefs in open-ended responses can therefore be informative for individual political competence as well as its crucial role in knowledge transmission through social interaction. 