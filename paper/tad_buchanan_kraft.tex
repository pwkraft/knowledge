\documentclass[12pt]{article}
\usepackage[margin = 1in]{geometry}
\usepackage[USenglish]{babel}
\usepackage{natbib}
\usepackage{graphicx}
\usepackage{fancyhdr}
\usepackage{setspace}
\usepackage{amsmath}
\usepackage{lscape}
\usepackage{dcolumn}
\usepackage{xcolor}
\usepackage[colorlinks=true,citecolor=red!50!black,urlcolor=blue!50!black,linkcolor=red!50!black]{hyperref}

\author{Laura Buchanan\footnote{\href{mailto:lcb402@nyu.edu}{lcb402@nyu.edu}} \and Patrick Kraft\footnote{\href{mailto:pk1622@nyu.edu}{pk1622@nyu.edu}}}
\date{today}

\title{Don't Just Ask Me for Facts!\\
\large{Measuring Political Sophistication using Open-Ended Responses}}
\date{\today}

% sans serif font
\renewcommand{\familydefault}{\sfdefault}


\begin{document}
\maketitle\doublespacing\thispagestyle{empty}

\begin{abstract}
There is a broad consensus among scholars of political science and public opinion that the American electorate is not well informed about politics. Interestingly however, there is no agreement in the discipline about \textit{how to measure} how little citizens actually know. While many studies rely on simple factual political knowledge questions to assess sophistication, others have criticized this approach from methodological and theoretical perspectives. We propose a new measure of political sophistication based on open-ended survey responses about individual preferences and evaluations of the most important problem facing the country. We presents results from the 2012 American National Election Study (ANES) and show that ...

\vspace{\baselineskip}
\noindent \textbf{Keywords:} political sophistication, measurement, open-ended responses, structural topic models \\

\noindent \textbf{Word Count:} ...
\end{abstract}
\newpage\setcounter{page}{1}



\section{Introduction}

On of the fundamental concepts in the study of political attitudes and behavior is political sophistication and knowledge \citep{converse1964nature,carpini1996americans}. While most scholars emphasized how little people know about politics, the question of how to assess individual knowledge has been subject to re-occurring scholarly debate \citep[e.g.][]{mondak2000reconsidering,mondak2001asked,sturgis2008experiment,debell2013harder,pietryka2013analysis}. Many analyses exclusively rely on individual levels of political information measured by factual knowledge questions. However, recent research points to important differences between types of knowledge questions that have previously been disregarded \citep{barabas2014question}. Furthermore, scholars argued that factual political knowledge as measured in many surveys may not be theoretically relevant \citep{lupia2006elitism} and the conceptualization of political sophistication should rather take into account how people structure their attitudes and beliefs \citep[e.g.][]{luskin1987measuring}. As such, measuring sophistication solely based on answers to political trivia may misclassify respondents who cannot recall these facts, but do indeed have a coherent framework of political ideas.

We propose an alternative measure of political sophistication based on individual responses to open-ended questions about attitudes towards major parties and presidential candidates. We make inferences about the respondents' level of political sophistication and belief constraint by focusing on \textit{how} respondents describe their preferences and beliefs. More specifically, we consider the diversity in topics raised by respondents based on structural topic models \citep{roberts2014structural} as well as other characteristics of individual open-ended responses, in order to assess the degree to which political attitudes are structured and expressed in a more complex manner. We suspect that the diversity in topics a respondent discusses, or the detail with which they speak about the topics they mention, will covary with other political knowledge measures. We therefore compare the text-based measures to common factual knowledge items as well as the interviewer assessment of the respondent's political knowledge as benchmarks.
% revise last part depending on evaluation method etc.

Overall, we hope to show that our measure of political sophistication can provide novel insights compared to conventional knowledge measures, since it is conceptually closer to the actual structure and constraint of political belief systems \citet[see for example][]{tetlock1983cognitive,luskin1987measuring}. Furthermore, developing valid measures of political sophistication based on open-ended responses will provide new opportunities for comparisons of political knowledge across time and contexts.


\section{Political Knowledge and Sophistication}
% 1: defining political participation in the literature
% 2: measuring political sophistication, previous approaches
% 3: issues with previous approaches, critiques

In his seminal study, \citet{converse1964nature} examined the degree to which citizens hold constrained belief systems about politics. In the paper, belief systems are defined as ``a configuration of ideas and attitudes in which the elements are bound together by some form of constraint or functional interdependence'' \citep[207]{converse1964nature}. The analyses showed that the majority of the electorate does not hold structured and constrained belief systems, understand abstract ideological concepts, or hold stable issue positions. 

This pessimistic finding regarding the competence of the US electorate has been confirmed in multiple subsequent analyses. \citet{carpini1996americans} showed that large parts of the American electorate are not sufficiently informed about politics. Furthermore, there are systematic differences in political attitudes and behavior between citizens who are well informed compared to those who are not. Such a finding is problematic from a normative perspectives, since it indicates that differences in levels of information can result in unequal representation in the political system \citep[see also][]{althaus1998information,kuklinski2000misinformation,gilens2001political}.

However, rather than relying on the degree to which individuals hold constrained belief systems, \citet{carpini1996americans}, but conceptualized political knowledge as awareness of key democratic values, which was measured using factual knowledge questions. A broad range of studies focused on similar measures of awareness and factual knowledge as indicators of sophistication \citep[e.g.][]{gomez2001political}. Most prominently, \citet{zaller1992nature}

However, even though most articles argue based on the structure of belief systems rather than basic knowledge, they just look at awareness. For example, Zaller...

Definition Zaller, political awareness, look at pieces of factual information, cite some studies that have relied on knowledge questions

But: there are issues with these knowledge questions, discuss Mondak, Prior etc.

Another example could be \citet{lupia1994shortcuts}, since it shows that factual knowledge might be less important as long as people can make inferences about positions etc. \citep[see also][]{lupia2006elitism}.

Also from a theoretical perspective, we might want to look at the structure of belief systems instead, discuss Luskin and Tetlock


\section{Open-Ended Responses}
% 1: how are open-ended responses administered
% 2: potential issues of open-ended responses
% 3: how can open-ended respones be beneficial? they are closer to the definition of political sophistication that we are interested in!




\section{Measurement Approach}
% 1: describe measure and how it relates to theoretical conceptualization of sophistication
% 2: derive validation of new measure

\section{Hypotheses and Validation Strategy}
% 1: correlation with other knowledge measures
% 2: replicate common findings
% 3: increase consistency b/w policy attitudes


\section{Data and Methods}

\section{Descriptive Results}

\section{Validation Performance}

\paragraph{Ideas for evaluation:}
\begin{itemize}\singlespacing
\item Increase in consistency b/w policy attitudes \citep[e.g.][]{prior2014visual}
\item replicate common findings, e.g. gender gap in political knowledge \citep[e.g.][]{barabas2014question}
\end{itemize}

\section{Discussion and Conclusion}


\clearpage
\bibliographystyle{apsr2006}
\bibliography{lit}


\end{document}