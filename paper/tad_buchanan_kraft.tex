\documentclass[12pt]{article}
\usepackage[margin = 1in]{geometry}
\usepackage[USenglish]{babel}
\usepackage{natbib}
\usepackage{graphicx}
\usepackage{fancyhdr}
\usepackage{setspace}
\usepackage{amsmath}
\usepackage{lscape}
\usepackage{dcolumn}
\usepackage{xcolor}
\usepackage[colorlinks=true,citecolor=red!50!black,urlcolor=blue!50!black,linkcolor=red!50!black]{hyperref}

\author{Laura Buchanan\footnote{\href{mailto:lcb402@nyu.edu}{lcb402@nyu.edu}} \and Patrick Kraft\footnote{\href{mailto:pk1622@nyu.edu}{pk1622@nyu.edu}}}
\date{today}

\title{Don't Just Ask Me for Facts!\\
\large{Measuring Political Sophistication using Open-Ended Responses}}
\date{\today}

% sans serif font
\renewcommand{\familydefault}{\sfdefault}


\begin{document}
\maketitle\doublespacing\thispagestyle{empty}

\begin{abstract}
There is a broad consensus among scholars of political science and public opinion that the American electorate is not well informed about politics. Interestingly however, there is no agreement in the discipline about \textit{how to measure} how little citizens actually know. While many studies rely on simple factual political knowledge questions to assess sophistication, others have criticized this approach from methodological and theoretical perspectives. We propose a new measure of political sophistication based on open-ended survey responses about individual preferences and evaluations of the most important problem facing the country. We presents results from the 2012 American National Election Study (ANES) and show that ...

\vspace{\baselineskip}
\noindent \textbf{Keywords:} political sophistication, measurement, open-ended responses, structural topic models \\

\noindent \textbf{Word Count:} ...
\end{abstract}
\newpage\setcounter{page}{1}



\section{Introduction}

On of the fundamental concepts in the study of political attitudes and behavior is political sophistication and knowledge \citep{converse1964nature,carpini1996americans}. While most scholars emphasized how little people know about politics, the question of how to assess individual knowledge has been subject to re-occurring scholarly debate \citep[e.g.][]{mondak2000reconsidering,mondak2001asked,sturgis2008experiment,debell2013harder,pietryka2013analysis}. Many analyses exclusively rely on individual levels of political information measured by factual knowledge questions. However, recent research points to important differences between types of knowledge questions that have previously been disregarded \citep{barabas2014question}. Furthermore, scholars argued that factual political knowledge as measured in many surveys may not be theoretically relevant \citep{lupia2006elitism} and the conceptualization of political sophistication should rather take into account how people structure their attitudes and beliefs \citep[e.g.][]{luskin1987measuring}. As such, measuring sophistication solely based on answers to political trivia may misclassify respondents who cannot recall these facts, but do indeed have a coherent framework of political ideas.

We propose an alternative measure of political sophistication based on individual responses to open-ended questions about attitudes towards major parties and presidential candidates. We make inferences about the respondents' level of political sophistication and belief constraint by focusing on \textit{how} respondents describe their preferences and beliefs. More specifically, we consider the diversity in topics raised by respondents based on structural topic models \citep{roberts2014structural} as well as other characteristics of individual open-ended responses, in order to assess the degree to which political attitudes are structured and expressed in a more complex manner. We suspect that the diversity in topics a respondent discusses, or the detail with which they speak about the topics they mention, will covary with other political knowledge measures. We therefore compare the text-based measures to common factual knowledge items as well as the interviewer assessment of the respondent's political knowledge as benchmarks.
% revise last part depending on evaluation method etc.

Overall, we hope to show that our measure of political sophistication can provide novel insights compared to conventional knowledge measures, since it is conceptually closer to the actual structure and constraint of political belief systems \citep[see for example][]{tetlock1983cognitive,luskin1987measuring}. Furthermore, developing valid measures of political sophistication based on open-ended responses will provide new opportunities for comparisons of political knowledge across time and contexts.


\section{Political Knowledge and Sophistication}
% 1: defining political participation in the literature
% 2: measuring political sophistication, previous approaches
% 3: issues with previous approaches, criticism

In his seminal study, \citet{converse1964nature} examined the degree to which citizens hold constrained belief systems about politics. In the paper, belief systems are defined as ``a configuration of ideas and attitudes in which the elements are bound together by some form of constraint or functional interdependence'' \citep[207]{converse1964nature}. The analyses showed that the majority of the electorate does not hold structured and constrained belief systems, understand abstract ideological concepts, or hold stable issue positions. 

This pessimistic view regarding the competence of the US electorate has been supported in multiple subsequent analyses. \citet{carpini1996americans} showed that large parts of the American electorate are not sufficiently informed about politics. Furthermore, there are systematic differences in political attitudes and behavior between citizens who are well informed compared to those who are not. Such a finding is problematic from a normative perspective, since it indicates that differences in levels of information can result in unequal representation in the political system \citep[see also][]{althaus1998information,kuklinski2000misinformation,gilens2001political}. However, rather than relying on the degree to which individuals hold constrained belief systems, \citet{carpini1996americans}, conceptualized knowledge as the awareness of key democratic values, which was measured using factual knowledge questions \citep[see also][]{carpini1993measuring}. A broad range of studies focused on similar factual knowledge measures as indicators of sophistication \citep[e.g.][]{zaller1991information,jacoby1995structure,gomez2001political}. Most prominently, \citet{zaller1992nature} argued for the measurement of political awareness using tests of neutral factual information about politics, since they ``more directly than any of the alternative measures, capture what has actually gotten into people’s minds'' \citep[21]{zaller1992nature}. However, other research casts doubt on this assertion, both from methodological as well as theoretical perspectives.

Methodologically, many studies raised issues related to the validity of factual knowledge questions. One fundamental problem discussed in the literature are potential biases due to guessing \citep{mondak2000reconsidering,mondak2001developing,mondak2001asked,miller2008experimenting}. Knowledge items that offer a ``Don't Know'' option essentially convolute two very distinct concepts: the individual information level as well as the propensity to guess. Based on this argument, \citet{mondak2004knowledge} showed that conventional knowledge measures overestimated the gender gap in political knowledge due to the fact that male respondents are more likely to take a guess if they are not fully informed \citep[see also][for a more recent discussion of differential item functioning as an explanation for knowledge gaps]{pietryka2013analysis}. The conclusions drawn from these studies were to rely on closed rather than open-ended knowledge questions and omitting ``Don't Know'' response options \citep[but see][]{sturgis2008experiment,luskin2011don}. Other scholars further criticized open-ended factual knowledge questions such as those administered in the American National Election Study due to problematic coding rules, which do not accurately capture partial knowledge \citep{krosnick2008problems,gibson2009knowing,debell2013harder}.

Focusing exclusively on factual political knowledge has also been criticized on theoretical grounds. For example, \citet{lupia2006elitism} argued that the information asked for in the item batteries has no clear relevance for individual political participation. Instead, researchers should concentrate on knowledge and heuristics that directly help citizens to make competent political decisions \citep[see also][]{lupia1994shortcuts}. Responses to factual knowledge questions have further been shown to be conditional on the respondents' motivation, their partisanship, as well as monetary incentives in the survey \citep{prior2008money,bullock2015partisan,prior2015you}. Conventional items also differ with regard to the specific dimension of political knowledge they measure \citep{barabas2014question} and ignore important aspects such as visual cues \citep{prior2014visual}.

Overall, the studies discussed so far suggest that the conventional item batteries have problematic measurement properties. More importantly, however, some authors raised doubts whether factual political knowledge actually captures the phenomena that are ultimately most interesting for scholars of public opinion. \citet{converse1964nature} initially discussed the level of constraint in political beliefs rather than isolated pieces of factual information about the political system. Other scholars emphasized similar conceptualizations of political sophistication. \citet{tetlock1983cognitive}, for example, used the term \textit{integrative complexity} to describe the variety and integration of considerations related to an issue. It is important to note that here, sophistication is not based on the content (or accuracy) of related considerations but rather on its \textit{structure}. \citet{luskin1987measuring} also defined political sophistication based on the structure of individual belief systems. More specifically, the author argues that belief systems can vary on three separate dimensions: (1) their \textit{size} of the -- i.e. the number of cognitions, (2) their \textit{range} -- i.e. the dispersion of cognition over categories, and (3) their \textit{constraint} -- i.e. the extent to which cognitions are interconnected. Political sophistication, in turn, is seen as the conjunction of these dimensions: ``A person is politically sophisticated to the extent to which his or her [political belief system] is large, wide-ranging, and highly constrained.'' \citep[860]{luskin1987measuring}.

Such a conceptualization of political sophistication seems theoretically more interesting and useful than simple tests of factual information. However, why does such a large body of literature in political science and public opinion then only focus on knowledge questions? One answer to this question is provided in the early study by \citet[206]{converse1964nature}, who stated: ``what is important to study cannot be measured and that what can be measured is not important to study.'' Factual political knowledge is much easier to assess (albeit not perfectly) than the structure of political belief systems. Indeed, \citet{tetlock1983cognitive} had to rely on manual coding of policy statements of US senators in order to assess their degree of integrative complexity. Such manual coding procedures, however, become increasingly infeasible with large amounts of text data (such as in large surveys). Recent advances in automated text analyses, on the other hand, provide us with the necessary tools to derive a measure of political sophistication that captures the theoretical arguments put forward by \citet{converse1964nature}, \citet{tetlock1983cognitive} and \citet{luskin1987measuring}, without the necessity of human coders. In the following, we will derive and explore such a measure based on open-ended survey responses.


\section{Measurement Approach}
% 1: describe measure and how it relates to theoretical conceptualization of sophistication (see Luskin's definition)
% 2: how are open-ended responses administered, potential issues
% 3: conclusion: coding open-ended responses gets us closer to the definition of political sophistication that we are actually interested in!


Tetlock used open-ended responses to measure integrative complexity. However, he relied on manual coding, which is very resource intensive. Our approach will capture the same concept but relying on automated coding procedures.



\section{Hypotheses and Validation Strategy}
% 1: correlation with other knowledge measures
% 2: replicate common findings
% 3: increase consistency b/w policy attitudes


\section{Data and Methods}

\section{Descriptive Results}
% show examples of high and low scoring political knowledge
% distribution of knowledge measure
% correlation plot with other knowledge measures (including individual aspects)

\section{Validation Performance}

\paragraph{Ideas for evaluation:}
\begin{itemize}\singlespacing
\item replicate common findings, e.g. gender gap in political knowledge \citep[e.g.][]{barabas2014question}
\item Increase in consistency b/w policy attitudes \citep[e.g.][]{prior2014visual}
\end{itemize}

\section{Discussion and Conclusion}


\clearpage\singlespacing\footnotesize
\bibliographystyle{apsr2006}
\bibliography{lit}


\end{document}