\documentclass[12pt]{article}
\usepackage[margin = 1in]{geometry}
\usepackage[USenglish]{babel}
\usepackage{natbib}
\usepackage{graphicx}
\usepackage{fancyhdr}
\usepackage{setspace}
\usepackage{amsmath}
\usepackage{lscape}
\usepackage{dcolumn}
\usepackage{xcolor}
\usepackage{longtable}
\usepackage{tabularx}
\usepackage{booktabs}
\usepackage{arydshln}
\usepackage{dcolumn}
\usepackage[colorlinks=true,citecolor=red!50!black,urlcolor=blue!50!black,linkcolor=red!50!black]{hyperref}

\author{Patrick W. Kraft\footnote{Ph.D. Candidate, Stony Brook University, \href{mailto:patrick.kraft@stonybrook.edu}{patrick.kraft@stonybrook.edu}.
%A previous version of this paper was in collaboration with Laura Buchanan, Data Science NYU
}}
\date{today}

\title{Women Also Know Stuff\\
\large{Challenging the Gender Gap in Political Sophistication}\footnote{Prepared for the 75th Annual Conference of the Midwest Political Science Association, April 6-9, 2017. The manuscript and code are available on GitHub: \url{https://github.com/pwkraft/knowledge}}
}
\date{\today}

% sans serif font
\renewcommand{\familydefault}{\sfdefault}


\begin{document}
\maketitle\doublespacing\thispagestyle{empty}

\begin{abstract}\singlespacing
Studies frequently found that on average, women appear to be less informed about politics than men. However, recent research raised theoretical as well as methodological concerns regarding conventional measures of political knowledge. This paper proposes an alternative approach to assess individual political sophistication in opinion surveys. Building on theoretical frameworks that focus on the structure of political belief systems rather than factual knowledge, I examine how individuals describe their political attitudes in open-ended responses using quantitative text analysis methods. The proposed measure aims to capture the complexity of verbatim responses based on their relative length, topic diversity, and opinion diversity. Compared to traditional knowledge metrics, the new measure behaves similarly as a predictor of political attitudes and behavior and shares common determinants - with one important exception. Contrary to previous research, there is no evidence for a gender gap in political sophistication using the text-based measure. While women might score lower than men on factual knowledge about political institutions and elites, there are no differences in the complexity of expressed political attitudes. %The paper proceeds to show how the new measure can improve our understanding of gender differences in political learning as well as the consequences of political sophistication. 

\vspace{\baselineskip}
\noindent \textbf{Keywords:} political sophistication, gender gap, measurement, open-ended responses, text analysis \\

\end{abstract}
\newpage\setcounter{page}{1}


\section*{Introduction}

% REVISE: change opening to discuss gender gap first
Political sophistication is one of the most fundamental concepts in the study of political attitudes and behavior. While scholars frequently emphasized the alarmingly low levels of political knowledge among the electorate \citep[e.g.,][]{converse1964nature,carpini1996americans}, there has been a re-occurring debate about how to assess individual sophistication accurately in the first place \citep[e.g.][]{mondak2000reconsidering,mondak2001asked,sturgis2008experiment,debell2013harder,pietryka2013analysis}. Many analyses exclusively rely on measures of factual knowledge about political institutions. However, recent research points to important differences between types of knowledge questions that have previously been disregarded \citep{barabas2014question}. Furthermore, scholars argue that factual political knowledge, as measured in many surveys, may not be theoretically relevant \citep{lupia2006elitism} and does not necessarily capture how people structure their attitudes and beliefs \citep[e.g.][]{luskin1987measuring}. Accordingly, measuring sophistication solely based on answers to political trivia may misclassify respondents who cannot recall these facts, but do indeed possess a coherent cognitive framework of political ideas.

One example for such potential misclassifications is the issue of gender differences in sophistication. On the basis of conventional factual knowledge scores, women frequently appear to be less informed about politics than men \citep{verba1997knowing,wolak2011roots}. However, some scholars suggested that these differences might be rooted in the conceptual and methodological issues related to the measurement approach. For example, \citet{mondak2004knowledge} argue that at least part of the gender gap in sophistication can be attributed to the fact that women are less likely to guess than men when facing a factual knowledge question for which they do not know an immediate answer. Others suggest that the gap can be attenuated by focusing on gender-relevant political knowledge \citep[e.g.,][]{dolan2011women} or by providing policy-specific information \citep[e.g.,][]{jerit2017revisiting}.

The present paper examines the gender gap by proposing an alternative measure of political sophistication based on the complexity of open-ended survey responses. As such, inferences about the respondents' level of political sophistication are based on how they describe their preferences and beliefs in their own words. For a given set of verbatim responses, the measure takes into account the relative response length, the diversity in topics raised by individuals, as well as their diversity in opinions. The goal is to assess whether political attitudes are expressed in a more complex manner --- a question that is not directly discernible from factual knowledge items. The text-based measure is therefore conceptually closer to the degree of structure and constraint in political belief systems
\citep[see for example][]{tetlock1983cognitive,luskin1987measuring}. 

Overall, the measure has similar characteristics as conventional factual knowledge scores. Indeed, text-based sophistication is a stronger predictor of internal efficacy, political engagement, and turnout than most traditional measures. Contrary to previous research, however, there is no evidence for a gender gap in political sophistication. While women might score lower than men on factual knowledge about political institutions and elites, there are no differences in the complexity of expressed political attitudes. More generally, the results suggest that developing valid measures of political sophistication based on open-ended responses can provide new opportunities to examine political knowledge across time and contexts. 


\section*{Political Knowledge and Sophistication}
% 1: gender gap is a commom phenomenon in the literature
% 2: potential explanations based on measurement
% 3: broader issue: do we need political facts or structure of belief systems? competence?

In his seminal study on political attitudes, \citet{converse1964nature} analyzed whether citizens hold constrained belief systems about politics. Belief systems are defined as ``a configuration of ideas and attitudes in which the elements are bound together by some form of constraint or functional interdependence'' \citep[207]{converse1964nature}. The author found that the majority of the electorate does not hold structured and constrained belief systems, understand abstract ideological concepts, or hold stable issue positions. This pessimistic view regarding the competence of the U.S. electorate has been supported in multiple subsequent analyses. \citet{carpini1996americans} show that large parts of the American electorate are not sufficiently informed about politics and that there are systematic differences in political attitudes and behavior between citizens who are well informed compared to those who are not. These findings are problematic from a normative perspective, since they indicate that differences in political information can result in unequal representation in the political system \citep[see also][]{althaus1998information,kuklinski2000misinformation,gilens2001political}. However, rather than relying on the degree to which individuals possess constrained belief systems, \citet{carpini1996americans} conceptualized knowledge as the awareness of key democratic values, using factual knowledge questions \citep[see also][]{carpini1993measuring}. Many studies focus on similar factual knowledge measures \citep[e.g.][]{zaller1991information,gomez2001political}. For example, \citet{zaller1992nature} measured political awareness using tests of neutral factual information about politics, since they ``more directly than any of the alternative measures, capture what has actually gotten into people’s minds'' \citep[21]{zaller1992nature}. However, other research casts doubt on this assertion, both from methodological as well as theoretical perspectives.

From a methodological perspective, many studies raised issues related to the validity of factual knowledge questions as a measure of political sophistication. One problem  are potential biases due to guessing \citep{mondak2000reconsidering,mondak2001developing,mondak2001asked,miller2008experimenting}. Responses to knowledge items that offer a ``Don't Know'' option are simultaneously influenced individual information levels as well as the propensity to guess. Other scholars further criticized open-ended factual knowledge questions due to problematic coding rules, which do not capture partial knowledge \citep{krosnick2008problems,gibson2009knowing,debell2013harder}.

Focusing solely on factual political knowledge has also been criticized on theoretical grounds. \citet{lupia2006elitism} argued that the information asked for has no clear relevance to political participation. Instead, researchers should concentrate on heuristics that directly help citizens to make competent political decisions or focus only on knowledge relevant to a specific task \citep[see also][]{lupia1994shortcuts}. Accordingly, there is no need for individuals to know all available facts, but only to possess the skills and resources to be able to \textit{find} the information required in a specific context \citep{prior2008money}. Furthermore, conventional items differ with regard to the dimension of political knowledge they measure \citep{barabas2014question} and ignore important aspects such as visual cues \citep{prior2014visual}.


\section*{The Gender Gap in Political Knowledge}

A common finding in research on political sophistication is the fact that women appear to be less knowledgeable about politics than men. For example, \citet{verba1997knowing} report that women score lower on political information, interest, and efficacy, which decreases their respective levels of political participation. Importantly, the authors argue that the gender differences can only partly be explained by resource-related factors such as individual levels of education. The authors ascribe the differences in political information and interest to a ``genuine difference in the taste for politics'' between men and women, which they suspect to be driven by socialization \citep[see also][]{wolak2011roots}.

Another explanation for the finding that disparities in resources (e.g., education) cannot fully account for gender differences is the fact that men and women benefit differently from the factors that increase political information \citep{dow2009gender}. As such, the gap is not only due to varying resource levels, but also due to differential gains from the resource itself. More broadly, this finding suggests that men and women consume political information through different channels \citep[see also][]{pietryka2013analysis}. Nevertheless, recent research showed that the gender gap can be substantially decreased given exposure to sufficient information \citep[e.g.][]{jerit2017revisiting} or through deliberation \citep{fraile2014does}.

Other scholars focused more closely on issues related to the measurement of political knowledge in order to explain the apparent gender gap. For example, \citet{mondak2004knowledge} suggest that women are more likely to report that they do not know the answer to a knowledge item if they are not completely certain, whereas men are more inclined to guess. Correcting for the systematic differences in the propensity to guess mitigates the gender gap in knowledge but does not eliminate it completely \citep[see also][]{lizotte2009explaining}. Based on their empirical evidence, \citet{mondak2004knowledge} elaborated on best practices regarding the measurement of political knowledge (e.g., using closed rather than open-ended knowledge items and discouraging `Don't Know' responses). Other related aspects of the survey context have also been shown to affect gender differences in political knowledge. For example, \citet{mcglone2006stereotype} present evidence that the gender gap is exacerbated in an environment that induces stereotype threat, for example if women are aware of the fact that the study focuses on gender differences or if they are interviewed by a male interviewer. However, gender differences are not only induced by \textit{how} researchers ask their questions, but also by the question \textit{content} itself. For example, \citet{dolan2011women} argues that the gap can be closed by focusing on gender-relevant political knowledge items such as information about women's representation in the federal government. Similarly, \citet{stolle2010women} report that the gender gap disappears when people are asked about more practical issues related to the government (e.g., benefits and services).

Overall, the gender gap has been shown to be influenced by how we ask for political information in surveys, as well as the kind of knowledge that is required for a correct response. Indeed, a comprehensive cross-national analysis of election studies in 47 countries between 1996 and 2011 suggests that question format and content account for large portions of the variance of gender disparities in political knowledge \citep{fortin2016cross}.


\section*{Measuring Sophistication in Open-Ended Survey Responses}
% TODO: merge this section and the following section (data and methods)
% 1: if we care about the structure of belief systems, how would they respond to open-ended items?
% 2: how are open-ended responses administered, potential issues

Many conventional item batteries to assess political sophistication have problematic measurement properties, which may exacerbate observed gender differences. Rather than trying to develop a new item battery that addresses some of these issues, I propose an alternative approach. Instead of testing respondents on a specific set of predetermined facts, we can make inferences about political sophistication by analyzing how they discuss their attitudes and beliefs in their own words.

Let's return to our initial discussion of the theoretical concept of interest -- political sophistication. As described above, \citet{converse1964nature} focused on the level of constraint in political beliefs rather than isolated pieces of factual information. Similarly, \citet{tetlock1983cognitive} used the term \textsl{integrative complexity} to describe the degree to which considerations related to an issue are interconnected. These studies do not conceptualize sophistication based on the content (or accuracy) of related considerations but rather on its \textsl{structure}. \citet{luskin1987measuring} also defined political sophistication based on the structure of individual belief systems, arguing that belief systems can vary on three separate dimensions: (1) their \textsl{size} -- i.e. the number of cognitions, (2) their \textsl{range} -- i.e. the dispersion of cognition over categories, and (3) their \textsl{constraint} -- i.e. the extent to which cognitions are interconnected in a meaningful way. Political sophistication, in turn, is seen as the conjunction of these dimensions: ``A person is politically sophisticated to the extent to which his or her [political belief system] is large, wide-ranging, and highly constrained.'' \citep[860]{luskin1987measuring}.

How would such a highly sophisticated person discuss his or her political beliefs as compared to a less politically sophisticated individual? Consider for example a survey where respondents are asked to describe what they like or dislike about different political parties and candidates in their own words. In such a scenario, the structure of individual political belief systems (i.e., size, range, and constraint) should be reflected in their verbatim responses to the open-ended items. In the following, I discuss three different attributes of open-ended survey responses that should be indicative of individual political sophistication as described by \citet{luskin1987measuring} and others.

First of all, sophisticated individuals should be able to elaborate more on their political attitudes. If people possess a large, wide-ranging, and constrained belief system, they should be able to recall a large number of considerations related to political parties and candidates. Verbatim responses of sophisticates can therefore be expected to have a greater overall \textbf{length}. 

However, sophisticated individuals should not only be able to talk about their attitudes at greater lengths. It is also important to consider the content of their respective answers. If a respondents holds more diverse cognitions towards political actors, we should observe a wider range of topics addressed in their responses rather than a focus on single issues. As such, verbatim responses of political sophisticates should display a greater degree of \textbf{topic diversity}.

Lastly, sophisticated individuals should hold opinions about each political actor and be able to express their attitudes towards each of them in terms of both, approval and disapproval. Responses that reflect high levels of sophistication should therefore display a greater level of \textbf{opinion diversity}.

Overall, a highly sophisticated individual can be expected to respond to a set of open-ended items by giving a more elaborate response that focuses on multiple political issues and addresses his or her attitudes towards all relevant political actors more or less equally. In the following, I derive and explore a text-based measure of political sophistication based on these characteristics of individual verbatim responses.


\section*{Data and Methods}
% describe dataset and open-ended responses
% describe measure for each dimension as well as composite measure of sophistication


The following analyses are based on the 2012 American National Election Study (ANES), which consists of a survey of 5914 adults. 2054 respondents participated in face-to-face interviews while the remaining 3860 filled out the survey online. For the purpose of the present analysis, I rely on the pooled dataset while controlling for differences in survey mode. 

The text-based sophistication measure is based on open-ended questions in which respondents were asked in the pre-election wave of the survey to list anything in particular that they like/dislike about the Democratic/Republican party as well as anything that might make them vote/not vote for either of the Presidential candidates. They were probed by the interviewer asking ``anything else?'' until the respondent answered ``no''. Overall, there are a total number of 8 open-ended responses where individuals described their beliefs and attitudes towards political actors. Verbatim responses were pre-processed using the Aspell spell-checking algorithm (\url{www.aspell.net}). Individuals who did not respond to all of the open-ended items (417 individuals), or who responded in Spanish (228 individuals), were excluded from the analysis. 

As discussed above, I consider three aspects of the open-ended responses to measure political sophistication in attitude expression: response length, topic diversity, and opinion diversity. 

The relative \textbf{length} individual responses is measured as the logged word count for each individual over all prompts:
\begin{equation}
\text{length}_i = \dfrac{\log\left(\sum_{j=1}^J n_{ij}\right)}{\max\left[\log\left(\sum_{j=1}^J n_{ij}\right)\right]},
\end{equation}
where $n_{ij}$ is the number of words in the response of individual $i$ in response to question $j$. $J$ denotes the set of all likes/dislikes items. I use the logged count to normalize the distribution of responses and divide it by the maximal response length in the data such that the resulting measure ranges from 0 to 1.

The \textbf{topic diversity} in individual responses is conceptualized as the relative mean absolute difference in topic proportions:\footnote{Individual topic proportions were extracted from a structural topic model estimated using the \texttt{stm} package in R \citep{roberts2014structural}. The number of topics was selected using the algorithm of \citet{lee2014low} and the model was estimated via spectral initialization to address the issue of multi-modality \citep[see][for details]{roberts2014stm}. I used measures for age, education, party identification, as well as an interaction between education and party identification as covariates for topic prevalence. This variable selection is equivalent to the procedure model specification described in \citet{roberts2014structural}. I estimated a total number of 72 topics. The results reported hereafter are robust for model specifications with fewer numbers of topics.}
\begin{equation}
\text{topic diversity}_i = 1-\dfrac{\sum_{k_1=1}^K\sum_{k_2=1}^K |\theta_{ik_1} - \theta_{ik_2}|}{2\sum_{k_1=1}^K\sum_{k_2=1}^K \theta_{ik_1}},
\end{equation}
where $\theta_{ik}$ denotes the predicted proportion of topic $k$ in the collection of responses by individual $i$. The variable ranges from 0 (response focuses on single topic), to 1 (every topic has the same proportion). Mathematically, this conceptualization is equivalent to the Gini-coefficient, which measures the degree of inequality in income distributions (although the direction has been reversed such that a value of 1 implies a perfectly equal distribution).

The last response characteristic, \textbf{opinion diversity}, is measured as the relative mean absolute difference of relative response lengths for each
likes/dislikes question:
\begin{equation}
\text{opinion diversity}_i = 1-\dfrac{\sum_{j_1=1}^J\sum_{j_2=1}^J |p_{ij_1} - p_{ij_2}|}{2\sum_{j_1=1}^J\sum_{j_2=1}^J p_{ij_1}},
\end{equation}
where $p_{ij}=\tfrac{n_{ij}}{\sum_{j=1}^J n_{ij}}$ is the proportion of words in the response of individual $i$ to question $j$ relative to the overall size of the individuals' response. Again, the variable ranges from 0 (only one question was answered) to 1 (all questions were answered with the same word length per answer).

Together, the three measures form a composite metric of political sophistication by calculating their respective average for each respondent. Like each individual component, the resulting \textbf{text-based sophistication} score ranges from 0 to 1:
\begin{equation}
\text{text-based sophistication}_i = \tfrac{1}{3}(\text{length}_i + \text{topic diversity}_i + \text{opinion diversity}_i).
\end{equation}

In the following, the text-based sophistication measure is compared to multiple alternative metrics of political knowledge. The most common way to measure political knowledge in surveys is to ask a set of factual questions about political institutions. The 2012 ANES includes such a basic item battery, inquiring for example about the number of times an individual can be elected President of the United States, or how the current U.S. federal budget deficit compares to the deficit in the 1990s. I combine individual responses on these items to a standard additive measure of \textbf{factual knowledge} about politics. The post-election wave of the 2012 ANES includes additional questions asking respondents to identify the offices of various politicians (e.g., John Boehner, Joe Biden) as well as to report which party controls the House of Representatives or the Senate after the election. Both sets of items are combined to additive indices for \textbf{office recognition} and knowledge about \textbf{majorities in Congress}, respectively. Additionally, the in-person sample of the 2012 ANES includes \textbf{interviewer assessments} of each respondent's political sophistication in the pre-election as well as the post-election wave.


\section*{Validating the Measure}

Figure~\ref{fig:corplot} provides a first validation of text-based sophistication by comparing it to the conventional knowledge metrics. The figure presents scatterplots between individual measures (lower triangular), univariate densities (diagonal), and correlation coefficients (upper triangular). The text-based sophistication measure is positively correlated with all conventional metrics while capturing some additional variation.

\begin{figure}[h]\centering
\includegraphics{../fig/corplot.pdf}
\caption{Correlation matrix of conventional political knowledge metrics and the text-based sophistication measure. The plots on the diagonal display univariate densities for each variable. The panels in the lower triangular display the scatter plot of two measures as well as a linear fit. The upper triangular displays the correlation coefficient. All correlations reported are statistically significant with $p<.05$.}\label{fig:corplot}
\end{figure}

Interestingly, for text-based sophistication, we observer the strongest correlation with interviewer evaluations in the pre-election wave. The text-based measure therefore captures characteristics that influence subjective assessments of sophistication. The interviewers certainly form their impressions throughout the entire survey, but it appears that the complexity of a respondent's verbatim answers is more influential than, for example, their performance on the factual knowledge questions.

Overall, while text-based sophistication and the alternative measures are clearly correlated, the relationship between each metrics is far from perfect. To provide some intuition whether the variation in text-based sophistication is theoretically meaningful, I present an example of open-ended responses of two individuals who scored equally on the factual knowledge score (3 out of 5 correct responses), but varied highly in text-based sophistication. The results are presented in Table~\ref{tab:ex1}.

\begin{table}[ht]\footnotesize\centering
\begin{tabular}{l|p{6.5cm}|p{6.5cm}}
   \toprule
  & A: Low Sophistication Response & B: High Sophistication Response \\ 
   \midrule
   Obama (+) & The healthcare, keeping that and the financial aid, helping students. & I think he is honest, has good intentions. \\ \hdashline
     Obama (-) &  & I don't feel he is up for the job, he doesn't really know how to get things accomplished from idea to actual reality. \\ \hdashline
     Romney (+) &  & He comes across as an honest person and I feel that financially he would be better for the country. \\ \hdashline
     Romney (-) & By taking financial aid away from students, taking family type planing, healthcare type of help away. & I am a moderate conservative and there are some things about anti-gay rights that I don't support. \\ \hdashline
     Democrats (+) & Mostly the healthcare, mostly people do need healthcare and can't afford to pay insurance. Financial aid most people cant afford to go college. Main two things that I like is the help with education and to pay for insurance to go to doctor. & They do seem to be generally concerned with everyone, taking care of the country as a whole. \\ \hdashline
     Democrats (-) &  & They fight too much among themselves and I disagree with wealth redistribution. \\ \hdashline
     Republicans (+) &  & I agree with a lot of the conservative values and taking responsibility for one's own actions. \\ \hdashline
     Republicans (-) &  & They argue too much among themselves and don't accomplish very much. \\ 
    \bottomrule
 \end{tabular}
\caption{Example of open-ended responses for low and high scores on the text-based sophistication measure with equal factual knowledge scores (3 out of 5 correct responses). Column A displays the verbatim responses of an individual who scored low on the text-based sophistication measure and column B displays the verbatim responses of an individual who scored high on the text-based sophistication measure. Each row represents one of the likes/dislikes items included in the analysis. Note that the responses in this table were slightly redacted for readability (spelling errors removed, etc.).}\label{tab:ex1}
\end{table}

Each row in the table represents one of the open-ended responses (like/dislike for each candidate/party). Column A displays the responses of an individual who scored low on text-based sophistication and column B displays the responses of a high scoring individual. Cells are empty if a respondent refused to provide a response. Even though both individuals received are measured to be equal in their factual political knowledge, there are systematic differences in their response behavior that can be attributed to their political sophistication. Overall, respondent A provided a less elaborate response, only focused on two issues (health care and student loans), and did not report attitudes on multiple items. Compared to B, such a response pattern is suggestive of a less sophisticated political belief system. This initial validation suggests that the variation in the sophistication measure captures meaningful differences in response behavior that clearly overlaps with traditional knowledge metrics while displaying some unique variation.

Political sophistication is not only of theoretical interest as a dependent variable in political science research, but commonly used as a determinant of other outcomes related to attitudes and behavior. As a last validation of text-based sophistication, Figure~\ref{fig:knoweff} presents the effects of each sophistication measure on four dependent variables commonly related to political sophistication: internal efficacy, external efficacy, non-conventional participation, and turnout. The results for the first three dependent variables are based on linear regressions while the effects on turnout are estimated using a logit model. Each model equation includes a single sophistication measure while controlling for gender, age, race, religiosity, and survey mode (face-to-face vs. online). Each plot displays the difference in the expected value of the respective dependent variable for maximum and minimum values of each sophistication measure, while holding all other variables at their means.

\begin{figure}[h]\centering
\includegraphics{../fig/knoweff.pdf}
\caption{Effects of sophistication on internal efficacy, external efficacy, non-conventional participation, and turnout. For each dependent variable, the figure displays the difference in expected values between maximum and minimum levels of sophistication observed on each measure (including 95\% confidence intervals). Model estimates are based on OLS (internal efficacy, external efficacy, non-conventional participation) or logistic regressions (turnout). Each sophistication measure is included in a single equation while controlling for gender, education, age, race, church attendance, and survey mode. Full model results ar presented in the appendix, Tables \ref{tab:inteff} through \ref{tab:turnout}}\label{fig:knoweff}
\end{figure}

Overall, the sophistication metrics perform similarly as predictors of internal efficacy, external efficacy, non-conventional participation, and turnout. The effect of text-based sophistication on the participation measures even appears to be stronger than the effect of any of the conventional metrics (except interviewer assessments).


\section*{Assessing the Gender Gap}
% APPENDIX: add additional analyses controlling for wordsum score, substantive results are unchanged

How do men and women compare on the different metrics of political sophistication? Figure~\ref{fig:meandiff} displays the average levels of text-based sophistication as well as the remaining metrics comparing both genders. While we observe a sizable gender gap for all five conventional political knowledge measures, the difference is substantially smaller for text-based sophistication. Here, the gender gap is still statistically significant, but substantively inconsequential when compared to the remaining measures.

\begin{figure}[h]\centering
\includegraphics{../fig/meandiff.pdf}
\caption{The gender gap in political sophistication. The figure displays mean levels of sophistication for each measure comparing men and women (including 95\% confidence intervals). The y-axis is scaled to range up to the maximum value observed in the data for each sophistication metric. All gender differences are statistically significant with $p<.05$.}\label{fig:meandiff}
\end{figure}


As described above, at least part of the gender gap can be attributed to real differences in resources relevant to political information (e.g., education). Accordingly, we need to control for common determinants of political knowledge across all available measures to provide a more comprehensive examination of potential gender differences. Previous studies consistently showed that political knowledge is positively affected by media exposure, frequent political discussions, and education. Furthermore, I include age, race, religiosity, and survey mode (face-to-face vs. online) as additional control variables. Figure~\ref{fig:determinants} displays the coefficients of regression models with each knowledge/sophistication measure as the dependent variable.

\begin{figure}[h]\centering
\includegraphics{../fig/determinants.pdf}
\caption{Common determinants of political sophistication. Estimates are OLS regression coefficients with 95\% confidence intervals. Dependent variables are the text-based sophistication measure as well as conventional metrics of political knowledge. Full model results are presented in the appendix, Table~\ref{tab:determinants}}\label{fig:determinants}
\end{figure}

After controlling for common determinants, text-based sophistication reveals no significant differences between men and women. On the other hand, we still observe the gender gap using all remaining political knowledge metrics considered here. As such, women might not score as highly on political quizzes (partly because they are less likely to guess rather express lack of knowledge), but they do not differ substantially in complexity and sophistication when they describe their political preferences.

The patterns for the remaining determinants are quite similar across different dependent variables. Knowledge and sophistication is significantly higher among respondents who are more exposed to political news media, discuss politics frequently, and are more educated. An interesting deviation, however, is the effect of survey mode. For factual knowledge questions, we observe that respondents in online surveys score significantly higher than individuals in face-to-face interviews. This difference could be explained by the fact that individuals are able to look up responses to factual knowledge questions while taking an online survey \citep[see also][]{clifford2016cheating}. For the text-based measure, on the other hand, we see that individuals appear to score lower on sophistication in online surveys. Respondents in online surveys therefore seem less willing to elaborate on their attitudes. Overall, the fact that the determinants of political sophistication are very consistent across models lends additional validity to the text-based measure.

As an additional step, we can examine whether women and men benefit equally from media exposure, political discussion, and education according to the distinct outcome measures. I extend the models from Figure~\ref{fig:determinants} by including interactions between gender and media exposure, gender and political discussion, as well as gender and education. For simplicity, I only focus on the comparison of text-based sophistication with the conventional factual knowledge measure. Figure~\ref{fig:closing} displays the expected sophistication levels for men and women on both metrics while increasing exposure, discussion, and education from their respective minimum to their maximum values (holding all other variables constant at their means).

\begin{figure}[h]\centering
\includegraphics{../fig/closing.pdf}
\caption{Closing the gender gap. The figure displays the expected values of text-based sophistication and factual knowledge for men and women depending on their respective levels of media exposure, political discussion, and education (including 95\% confidence intervals). Estimates are based on OLS including interactions between gender and each independent variable. All models additionally control for education, age, race, church attendance, and survey mode. Full model results are presented in the appendix, Table~\ref{tab:closing}.}\label{fig:closing}
\end{figure}

Media exposure, political discussions, and education increase political sophistication among men and women -- irrespective of the specific measurement. However, there are important differences between the text-based measure and factual knowledge. For the conventional metric, we observe that the gender gap is not reduced with higher levels of media exposure, political discussion, or education. Indeed, media exposure even appears to increase gender differences in factual knowledge. This result seems inconsistent with recent research indicating that information exposure can mitigate the gender gap \citep[e.g.,][]{jerit2017revisiting}. However, the growing difference in factual knowledge between men and women due to more frequent media exposure might be explained by the fact that the item battery does not capture gender-relevant \citep{dolan2011women} or more practical \citep{stolle2010women} political knowledge. For the text-based sophistication measure, on the other hand, women benefit at least as much as men (if not more) from media exposure, political discussion, and education. Overall, solely focusing on factual knowledge may therefore lead to inaccurate conclusions regarding potential ways to mitigate gender differences in political sophistication.
% JENN: not satisfying


\section*{Conclusion}
% lot's of potential extensions, think about standardizing the measure to make it more comparable across contexts

As Arthur Lupia \citeyearpar{lupia2015uninformed} described in his recent book ``Uninformed'', political scientists should worry less about pure levels of \textit{information}, but rather focus on the necessary conditions for individuals to make \textit{competent} decisions. Competence in the context of political decision-making and voting requires citizens to hold informed attitudes about their representatives. Factual knowledge about political institutions might be a useful proxy for competence in certain scenarios. However, it cannot address directly whether individuals are sufficiently opinionated about political actors they try to hold accountable. In comparison, the text-based sophistication measure proposed here is agnostic about the specific contents of individual beliefs, but directly captures the structure and complexity in attitude expression.

The findings presented in this paper show that conventional knowledge indices and the text-based measure share a substantial amount of variance. However, they are far from being identical and capture different aspects of sophistication. Most importantly, using the text-based measure, any evidence for the gender gap commonly reported using factual knowledge scales disappears. Women might know fewer facts about political institutions, but they do not differ substantively in the complexity of their expressed political beliefs.


\singlespacing
\bibliographystyle{/data/Dropbox/Uni/Lit/apsr2006}
\bibliography{/data/Dropbox/Uni/Lit/Literature}

\clearpage
\section*{Appendix A: Details on Open-ended Responses}
\renewcommand\thefigure{A.\arabic{figure}}
\renewcommand\thetable{A.\arabic{table}}
\setcounter{figure}{0}
\setcounter{table}{0}

\begin{figure}[h]\centering
\includegraphics{../fig/wc.pdf}
\caption{Histogram of total word count in the collection of open-ended responses for each individual (left panel) and distribution of logged word count used in the text-based sophistication measure (right panel, re-scaled to range from 0 to 1). Dashed red lines indicate mean values. Most respondents provide brief statements when they describe their attitudes towards political parties and candidates. The mean response length to all 8 questions is about 75 words, so an average response to a single question consisted of less than 10 words, omitting respondents who did not provide any information.}\label{fig:wc}
\end{figure}

\begin{figure}[h]\centering
\includegraphics{../fig/diversity.pdf}
\caption{Histogram and density of the topic diversity (left panel) and opinion diversity (right panel) measure. Dashed red lines indicate mean values. The spike at 0 for opinion diversity are due to the fact that a large proportion of respondents only answered a single open-ended question.}\label{fig:diversity}
\end{figure}


\clearpage
\section*{Appendix B: Tables of Model Estimates}
\renewcommand\thefigure{B.\arabic{figure}}
\renewcommand\thetable{B.\arabic{table}}
\setcounter{figure}{0}
\setcounter{table}{0}


% Table created by stargazer v.5.2 by Marek Hlavac, Harvard University. E-mail: hlavac at fas.harvard.edu
% Date and time: Mon, Apr 03, 2017 - 04:16:03 PM
% Requires LaTeX packages: dcolumn 
\begin{table}[ht] \centering 
  \caption{Effects of sophistication -- OLS models predicting internal efficacy 
          based on different sophistication 
          measures. Positive coefficients indicate higher self-reported internal efficacy. 
          Standard errors in parentheses. Estimates are used for Figure~\ref{fig:knoweff} 
          in the main text.} 
  \label{tab:inteff} 
\scriptsize 
\begin{tabular}{@{\extracolsep{-5pt}}lD{.}{.}{-3} D{.}{.}{-3} D{.}{.}{-3} D{.}{.}{-3} D{.}{.}{-3} D{.}{.}{-3} } 
\\[-1.8ex]\hline 
\hline \\[-1.8ex] 
 & \multicolumn{6}{c}{\textit{Dependent variable:}} \\ 
\cline{2-7} 
\\[-1.8ex] & \multicolumn{6}{c}{Iternal Efficacy} \\ 
\\[-1.8ex] & \multicolumn{1}{c}{(1)} & \multicolumn{1}{c}{(2)} & \multicolumn{1}{c}{(3)} & \multicolumn{1}{c}{(4)} & \multicolumn{1}{c}{(5)} & \multicolumn{1}{c}{(6)}\\ 
\hline \\[-1.8ex] 
 Text-based & 0.554^{***} &  &  &  &  &  \\ 
  & (0.035) &  &  &  &  &  \\ 
  Factual &  & 0.237^{***} &  &  &  &  \\ 
  &  & (0.015) &  &  &  &  \\ 
  Office &  &  & 0.248^{***} &  &  &  \\ 
  &  &  & (0.011) &  &  &  \\ 
  Majorities &  &  &  & 0.140^{***} &  &  \\ 
  &  &  &  & (0.008) &  &  \\ 
  Eval. (Pre) &  &  &  &  & 0.361^{***} &  \\ 
  &  &  &  &  & (0.019) &  \\ 
  Eval. (Post) &  &  &  &  &  & 0.285^{***} \\ 
  &  &  &  &  &  & (0.020) \\ 
  Sex (Female) & -0.062^{***} & -0.050^{***} & -0.052^{***} & -0.053^{***} & -0.038^{***} & -0.040^{***} \\ 
  & (0.006) & (0.006) & (0.006) & (0.006) & (0.010) & (0.010) \\ 
  Education (College) & 0.056^{***} & 0.055^{***} & 0.038^{***} & 0.063^{***} & 0.035^{**} & 0.044^{***} \\ 
  & (0.006) & (0.006) & (0.007) & (0.006) & (0.011) & (0.012) \\ 
  log(Age) & 0.019^{*} & 0.008 & 0.009 & 0.010 & -0.016 & -0.005 \\ 
  & (0.008) & (0.008) & (0.008) & (0.008) & (0.012) & (0.013) \\ 
  Race (Black) & 0.041^{***} & 0.058^{***} & 0.048^{***} & 0.042^{***} & 0.050^{***} & 0.059^{***} \\ 
  & (0.008) & (0.008) & (0.008) & (0.008) & (0.011) & (0.011) \\ 
  Church Attendance & 0.005 & 0.008 & 0.013 & 0.005 & -0.003 & -0.009 \\ 
  & (0.008) & (0.008) & (0.008) & (0.009) & (0.014) & (0.015) \\ 
  Survey Mode (Online) & 0.047^{***} & 0.009 & -0.009 & 0.001 &  &  \\ 
  & (0.006) & (0.006) & (0.007) & (0.007) &  &  \\ 
  Constant & 0.179^{***} & 0.368^{***} & 0.442^{***} & 0.444^{***} & 0.377^{***} & 0.400^{***} \\ 
  & (0.033) & (0.030) & (0.030) & (0.031) & (0.044) & (0.047) \\ 
 \hline \\[-1.8ex] 
Observations & \multicolumn{1}{c}{5,135} & \multicolumn{1}{c}{5,135} & \multicolumn{1}{c}{4,816} & \multicolumn{1}{c}{4,816} & \multicolumn{1}{c}{1,743} & \multicolumn{1}{c}{1,647} \\ 
R$^{2}$ & \multicolumn{1}{c}{0.114} & \multicolumn{1}{c}{0.116} & \multicolumn{1}{c}{0.160} & \multicolumn{1}{c}{0.128} & \multicolumn{1}{c}{0.228} & \multicolumn{1}{c}{0.170} \\ 
\hline 
\hline \\[-1.8ex] 
\textit{Note:}  & \multicolumn{6}{r}{$^{*}$p$<$0.05; $^{**}$p$<$0.01; $^{***}$p$<$0.001} \\ 
\end{tabular} 
\end{table} 


% Table created by stargazer v.5.2 by Marek Hlavac, Harvard University. E-mail: hlavac at fas.harvard.edu
% Date and time: Fri, Mar 31, 2017 - 11:40:42 PM
% Requires LaTeX packages: dcolumn 
\begin{table}[ht] \centering 
  \caption{Effects on External Efficacy} 
  \label{tab:exteff} 
\scriptsize 
\begin{tabular}{@{\extracolsep{-5pt}}lD{.}{.}{-3} D{.}{.}{-3} D{.}{.}{-3} D{.}{.}{-3} D{.}{.}{-3} D{.}{.}{-3} } 
\\[-1.8ex]\hline 
\hline \\[-1.8ex] 
 & \multicolumn{6}{c}{\textit{Dependent variable:}} \\ 
\cline{2-7} 
\\[-1.8ex] & \multicolumn{6}{c}{External Efficacy} \\ 
\hline \\[-1.8ex] 
 Text-based & 0.130^{**} &  &  &  &  &  \\ 
  & (0.040) &  &  &  &  &  \\ 
  Factual &  & 0.059^{***} &  &  &  &  \\ 
  &  & (0.017) &  &  &  &  \\ 
  Office &  &  & 0.089^{***} &  &  &  \\ 
  &  &  & (0.013) &  &  &  \\ 
  Majorities &  &  &  & 0.045^{***} &  &  \\ 
  &  &  &  & (0.009) &  &  \\ 
  Eval. (Pre) &  &  &  &  & 0.139^{***} &  \\ 
  &  &  &  &  & (0.025) &  \\ 
  Eval. (Post) &  &  &  &  &  & 0.147^{***} \\ 
  &  &  &  &  &  & (0.026) \\ 
  Sex (Female) & 0.015^{*} & 0.018^{**} & 0.017^{*} & 0.016^{*} & 0.026^{*} & 0.025 \\ 
  & (0.007) & (0.007) & (0.007) & (0.007) & (0.013) & (0.013) \\ 
  Education (College) & 0.045^{***} & 0.045^{***} & 0.035^{***} & 0.045^{***} & 0.064^{***} & 0.057^{***} \\ 
  & (0.007) & (0.007) & (0.008) & (0.007) & (0.015) & (0.015) \\ 
  log(Age) & -0.006 & -0.009 & -0.013 & -0.012 & -0.052^{**} & -0.050^{**} \\ 
  & (0.009) & (0.009) & (0.009) & (0.009) & (0.016) & (0.017) \\ 
  Race (Black) & 0.073^{***} & 0.077^{***} & 0.076^{***} & 0.074^{***} & 0.052^{***} & 0.056^{***} \\ 
  & (0.009) & (0.009) & (0.009) & (0.009) & (0.014) & (0.015) \\ 
  Church Attendance & 0.050^{***} & 0.051^{***} & 0.055^{***} & 0.052^{***} & 0.049^{**} & 0.046^{*} \\ 
  & (0.009) & (0.009) & (0.010) & (0.010) & (0.019) & (0.019) \\ 
  Survey Mode (Online) & -0.034^{***} & -0.043^{***} & -0.056^{***} & -0.052^{***} &  &  \\ 
  & (0.007) & (0.007) & (0.008) & (0.008) &  &  \\ 
  Constant & 0.358^{***} & 0.403^{***} & 0.440^{***} & 0.439^{***} & 0.464^{***} & 0.467^{***} \\ 
  & (0.038) & (0.035) & (0.036) & (0.036) & (0.059) & (0.061) \\ 
 \hline \\[-1.8ex] 
Observations & \multicolumn{1}{c}{5,123} & \multicolumn{1}{c}{5,123} & \multicolumn{1}{c}{4,803} & \multicolumn{1}{c}{4,803} & \multicolumn{1}{c}{1,732} & \multicolumn{1}{c}{1,634} \\ 
R$^{2}$ & \multicolumn{1}{c}{0.040} & \multicolumn{1}{c}{0.041} & \multicolumn{1}{c}{0.049} & \multicolumn{1}{c}{0.045} & \multicolumn{1}{c}{0.054} & \multicolumn{1}{c}{0.053} \\ 
\hline 
\hline \\[-1.8ex] 
\textit{Note:}  & \multicolumn{6}{r}{$^{*}$p$<$0.05; $^{**}$p$<$0.01; $^{***}$p$<$0.001} \\ 
\end{tabular} 
\end{table} 


% Table created by stargazer v.5.2 by Marek Hlavac, Harvard University. E-mail: hlavac at fas.harvard.edu
% Date and time: Thu, Jun 22, 2017 - 04:21:46 PM
% Requires LaTeX packages: dcolumn 
\begin{table}[ht] \centering 
  \caption{Effects of sophistication -- OLS models predicting non-conventional 
          particpation (protest, signing 
          petitions, etc.) based on different sophistication 
          measures. Positive coefficients indicate higher levels of participation. 
          Standard errors in parentheses. Estimates are used for Figure~\ref{fig:knoweff} 
          in the main text.} 
  \label{tab:nonconv} 
\scriptsize 
\begin{tabular}{@{\extracolsep{-5pt}}lD{.}{.}{-3} D{.}{.}{-3} D{.}{.}{-3} D{.}{.}{-3} D{.}{.}{-3} D{.}{.}{-3} } 
\\[-1.8ex]\hline 
\hline \\[-1.8ex] 
 & \multicolumn{6}{c}{\textit{Dependent variable:}} \\ 
\cline{2-7} 
\\[-1.8ex] & \multicolumn{6}{c}{Non-conventional Participation} \\ 
\hline \\[-1.8ex] 
 Text-based & 1.124^{***} &  &  &  &  &  \\ 
  & (0.085) &  &  &  &  &  \\ 
  Factual &  & 0.193^{***} &  &  &  &  \\ 
  &  & (0.037) &  &  &  &  \\ 
  Office &  &  & 0.322^{***} &  &  &  \\ 
  &  &  & (0.028) &  &  &  \\ 
  Majorities &  &  &  & 0.151^{***} &  &  \\ 
  &  &  &  & (0.020) &  &  \\ 
  Eval. (Pre) &  &  &  &  & 0.453^{***} &  \\ 
  &  &  &  &  & (0.050) &  \\ 
  Eval. (Post) &  &  &  &  &  & 0.451^{***} \\ 
  &  &  &  &  &  & (0.050) \\ 
  Sex (Female) & 0.010 & 0.017 & 0.025 & 0.021 & 0.033 & 0.036 \\ 
  & (0.014) & (0.015) & (0.014) & (0.015) & (0.025) & (0.025) \\ 
  Education (College) & 0.082^{***} & 0.110^{***} & 0.082^{***} & 0.111^{***} & 0.103^{***} & 0.096^{**} \\ 
  & (0.017) & (0.017) & (0.017) & (0.016) & (0.030) & (0.030) \\ 
  Income & 0.063^{*} & 0.082^{**} & 0.048 & 0.087^{***} & 0.036 & 0.040 \\ 
  & (0.026) & (0.027) & (0.026) & (0.026) & (0.046) & (0.046) \\ 
  log(Age) & 0.036 & 0.038 & 0.026 & 0.031 & -0.057 & -0.054 \\ 
  & (0.019) & (0.020) & (0.019) & (0.020) & (0.031) & (0.031) \\ 
  Race (Black) & 0.040^{*} & 0.048^{*} & 0.042^{*} & 0.037 & -0.020 & -0.012 \\ 
  & (0.019) & (0.019) & (0.019) & (0.019) & (0.028) & (0.028) \\ 
  Church Attendance & 0.025 & 0.032 & 0.038 & 0.028 & 0.046 & 0.041 \\ 
  & (0.020) & (0.021) & (0.020) & (0.020) & (0.036) & (0.036) \\ 
  Survey Mode (Online) & 0.105^{***} & 0.055^{***} & 0.024 & 0.041^{*} &  &  \\ 
  & (0.016) & (0.016) & (0.016) & (0.016) &  &  \\ 
  Constant & -0.415^{***} & 0.008 & 0.108 & 0.092 & 0.270^{*} & 0.284^{*} \\ 
  & (0.081) & (0.075) & (0.075) & (0.075) & (0.115) & (0.115) \\ 
 \hline \\[-1.8ex] 
Observations & \multicolumn{1}{c}{4,701} & \multicolumn{1}{c}{4,701} & \multicolumn{1}{c}{4,701} & \multicolumn{1}{c}{4,701} & \multicolumn{1}{c}{1,573} & \multicolumn{1}{c}{1,573} \\ 
R$^{2}$ & \multicolumn{1}{c}{0.071} & \multicolumn{1}{c}{0.042} & \multicolumn{1}{c}{0.063} & \multicolumn{1}{c}{0.048} & \multicolumn{1}{c}{0.083} & \multicolumn{1}{c}{0.082} \\ 
\hline 
\hline \\[-1.8ex] 
\textit{Note:}  & \multicolumn{6}{r}{$^{*}$p$<$0.05; $^{**}$p$<$0.01; $^{***}$p$<$0.001} \\ 
\end{tabular} 
\end{table} 


% Table created by stargazer v.5.2 by Marek Hlavac, Harvard University. E-mail: hlavac at fas.harvard.edu
% Date and time: Wed, Jun 14, 2017 - 11:02:17 AM
% Requires LaTeX packages: dcolumn 
\begin{table}[ht] \centering 
  \caption{Effects of sophistication -- Logit models predicting turnout based on 
          different sophistication measures. Positive coefficients indicate higher 
          probabilities to participate in the election. 
          Standard errors in parentheses. Estimates are used for Figure~\ref{fig:knoweff} 
          in the main text.} 
  \label{tab:turnout} 
\scriptsize 
\begin{tabular}{@{\extracolsep{-5pt}}lD{.}{.}{-3} D{.}{.}{-3} D{.}{.}{-3} D{.}{.}{-3} D{.}{.}{-3} D{.}{.}{-3} } 
\\[-1.8ex]\hline 
\hline \\[-1.8ex] 
 & \multicolumn{6}{c}{\textit{Dependent variable:}} \\ 
\cline{2-7} 
\\[-1.8ex] & \multicolumn{6}{c}{Turnout} \\ 
\hline \\[-1.8ex] 
 Text-based & 5.082^{***} &  &  &  &  &  \\ 
  & (0.504) &  &  &  &  &  \\ 
  Factual &  & 1.087^{***} &  &  &  &  \\ 
  &  & (0.201) &  &  &  &  \\ 
  Office &  &  & 1.982^{***} &  &  &  \\ 
  &  &  & (0.183) &  &  &  \\ 
  Majorities &  &  &  & 1.089^{***} &  &  \\ 
  &  &  &  & (0.110) &  &  \\ 
  Eval. (Pre) &  &  &  &  & 2.795^{***} &  \\ 
  &  &  &  &  & (0.282) &  \\ 
  Eval. (Post) &  &  &  &  &  & 2.726^{***} \\ 
  &  &  &  &  &  & (0.285) \\ 
  Sex (Female) & 0.070 & 0.072 & 0.135 & 0.135 & 0.172 & 0.190 \\ 
  & (0.083) & (0.083) & (0.084) & (0.084) & (0.134) & (0.135) \\ 
  Education (College) & 0.503^{***} & 0.604^{***} & 0.438^{***} & 0.585^{***} & 0.494^{**} & 0.438^{*} \\ 
  & (0.108) & (0.107) & (0.109) & (0.107) & (0.185) & (0.184) \\ 
  Income & 1.076^{***} & 1.116^{***} & 0.949^{***} & 1.164^{***} & 1.107^{***} & 1.168^{***} \\ 
  & (0.152) & (0.152) & (0.155) & (0.152) & (0.259) & (0.257) \\ 
  log(Age) & 1.052^{***} & 1.035^{***} & 0.974^{***} & 0.974^{***} & 0.726^{***} & 0.703^{***} \\ 
  & (0.103) & (0.103) & (0.104) & (0.104) & (0.160) & (0.160) \\ 
  Race (Black) & 0.816^{***} & 0.864^{***} & 0.823^{***} & 0.804^{***} & 0.943^{***} & 0.979^{***} \\ 
  & (0.119) & (0.119) & (0.118) & (0.118) & (0.165) & (0.165) \\ 
  Church Attendance & 0.705^{***} & 0.733^{***} & 0.761^{***} & 0.714^{***} & 0.813^{***} & 0.757^{***} \\ 
  & (0.127) & (0.126) & (0.127) & (0.127) & (0.205) & (0.205) \\ 
  Survey Mode (Online) & 0.643^{***} & 0.388^{***} & 0.218^{*} & 0.236^{**} &  &  \\ 
  & (0.088) & (0.087) & (0.089) & (0.090) &  &  \\ 
  Constant & -6.639^{***} & -4.709^{***} & -4.214^{***} & -4.183^{***} & -4.265^{***} & -3.990^{***} \\ 
  & (0.449) & (0.391) & (0.395) & (0.395) & (0.604) & (0.598) \\ 
 \hline \\[-1.8ex] 
Observations & \multicolumn{1}{c}{4,724} & \multicolumn{1}{c}{4,724} & \multicolumn{1}{c}{4,706} & \multicolumn{1}{c}{4,706} & \multicolumn{1}{c}{1,582} & \multicolumn{1}{c}{1,580} \\ 
Akaike Inf. Crit. & \multicolumn{1}{c}{3,809.757} & \multicolumn{1}{c}{3,885.718} & \multicolumn{1}{c}{3,777.952} & \multicolumn{1}{c}{3,809.053} & \multicolumn{1}{c}{1,470.846} & \multicolumn{1}{c}{1,477.271} \\ 
\hline 
\hline \\[-1.8ex] 
\textit{Note:}  & \multicolumn{6}{r}{$^{*}$p$<$0.05; $^{**}$p$<$0.01; $^{***}$p$<$0.001} \\ 
\end{tabular} 
\end{table} 


% Table created by stargazer v.5.2 by Marek Hlavac, Harvard University. E-mail: hlavac at fas.harvard.edu
% Date and time: Wed, Jun 14, 2017 - 11:02:19 AM
% Requires LaTeX packages: dcolumn 
\begin{table}[ht] \centering 
  \caption{Determinants of political knowledge -- OLS models predicting different 
          political sophistication measures.
          Positive coefficients indicate higher sophistication. 
          Standard errors in parentheses. Estimates are used for Figure~\ref{fig:determinants} 
          in the main text.} 
  \label{tab:determinants} 
\scriptsize 
\begin{tabular}{@{\extracolsep{-5pt}}lD{.}{.}{-3} D{.}{.}{-3} D{.}{.}{-3} D{.}{.}{-3} D{.}{.}{-3} D{.}{.}{-3} } 
\\[-1.8ex]\hline 
\hline \\[-1.8ex] 
 & \multicolumn{6}{c}{Dependent Variable: Political Knowledge Measure} \\ 
\cline{2-7} 
\\[-1.8ex] & \multicolumn{1}{c}{Text-based} & \multicolumn{1}{c}{Factual} & \multicolumn{1}{c}{Office} & \multicolumn{1}{c}{Majorities} & \multicolumn{1}{c}{Eval. (Pre)} & \multicolumn{1}{c}{Eval. (Post)} \\ 
\hline \\[-1.8ex] 
 Gender (Female) & -0.002 & -0.049^{***} & -0.045^{***} & -0.082^{***} & -0.046^{***} & -0.052^{***} \\ 
  & (0.002) & (0.006) & (0.007) & (0.011) & (0.012) & (0.012) \\ 
  Media Exposure & 0.024^{***} & 0.069^{***} & 0.159^{***} & 0.193^{***} & 0.255^{***} & 0.192^{***} \\ 
  & (0.006) & (0.014) & (0.018) & (0.026) & (0.030) & (0.030) \\ 
  Political Discussions & 0.056^{***} & 0.062^{***} & 0.195^{***} & 0.200^{***} & 0.144^{***} & 0.174^{***} \\ 
  & (0.004) & (0.010) & (0.013) & (0.019) & (0.019) & (0.019) \\ 
  Education (College) & 0.038^{***} & 0.091^{***} & 0.129^{***} & 0.099^{***} & 0.100^{***} & 0.118^{***} \\ 
  & (0.003) & (0.007) & (0.008) & (0.012) & (0.014) & (0.014) \\ 
  Income & 0.031^{***} & 0.117^{***} & 0.147^{***} & 0.091^{***} & 0.142^{***} & 0.135^{***} \\ 
  & (0.004) & (0.010) & (0.013) & (0.019) & (0.022) & (0.022) \\ 
  log(Age) & 0.012^{***} & 0.081^{***} & 0.048^{***} & 0.122^{***} & 0.057^{***} & 0.062^{***} \\ 
  & (0.003) & (0.008) & (0.010) & (0.015) & (0.015) & (0.015) \\ 
  Race (Black) & -0.009^{**} & -0.085^{***} & -0.036^{***} & -0.034^{*} & 0.031^{*} & 0.020 \\ 
  & (0.003) & (0.008) & (0.010) & (0.014) & (0.014) & (0.014) \\ 
  Church Attendance & 0.004 & -0.004 & -0.028^{**} & 0.014 & 0.003 & 0.014 \\ 
  & (0.003) & (0.008) & (0.010) & (0.015) & (0.017) & (0.017) \\ 
  Survey Mode (Online) & -0.024^{***} & 0.093^{***} & 0.163^{***} & 0.214^{***} &  &  \\ 
  & (0.003) & (0.006) & (0.008) & (0.012) &  &  \\ 
  Constant & 0.389^{***} & 0.101^{***} & -0.198^{***} & -0.376^{***} & 0.199^{***} & 0.137^{*} \\ 
  & (0.012) & (0.030) & (0.038) & (0.055) & (0.056) & (0.055) \\ 
 \hline \\[-1.8ex] 
Observations & \multicolumn{1}{c}{4,698} & \multicolumn{1}{c}{4,698} & \multicolumn{1}{c}{4,698} & \multicolumn{1}{c}{4,698} & \multicolumn{1}{c}{1,575} & \multicolumn{1}{c}{1,575} \\ 
R$^{2}$ & \multicolumn{1}{c}{0.159} & \multicolumn{1}{c}{0.282} & \multicolumn{1}{c}{0.322} & \multicolumn{1}{c}{0.225} & \multicolumn{1}{c}{0.242} & \multicolumn{1}{c}{0.251} \\ 
\hline 
\hline \\[-1.8ex] 
\textit{Note:}  & \multicolumn{6}{r}{$^{*}$p$<$0.05; $^{**}$p$<$0.01; $^{***}$p$<$0.001} \\ 
\end{tabular} 
\end{table} 


% Table created by stargazer v.5.2 by Marek Hlavac, Harvard University. E-mail: hlavac at fas.harvard.edu
% Date and time: Mon, Apr 03, 2017 - 04:16:11 PM
% Requires LaTeX packages: dcolumn 
\begin{table}[ht] \centering 
  \caption{Closing the Gender Gap -- OLS models predicting different 
          political sophistication measures.
          Positive coefficients indicate higher sophistication. 
          Standard errors in parentheses. Estimates are used for 
          Figure~\ref{fig:closing} in the main text.} 
  \label{tab:closing} 
\scriptsize 
\begin{tabular}{@{\extracolsep{-5pt}}lD{.}{.}{-3} D{.}{.}{-3} } 
\\[-1.8ex]\hline 
\hline \\[-1.8ex] 
 & \multicolumn{2}{c}{Dependent Variable: Political Knowledge Measure} \\ 
\cline{2-3} 
\\[-1.8ex] & \multicolumn{1}{c}{Text-based} & \multicolumn{1}{c}{Factual} \\ 
\hline \\[-1.8ex] 
 Gender (Female) & -0.012^{*} & -0.023 \\ 
  & (0.005) & (0.013) \\ 
  Media Exposure & 0.021^{**} & 0.112^{***} \\ 
  & (0.008) & (0.020) \\ 
  Political Discussions & 0.057^{***} & 0.074^{***} \\ 
  & (0.006) & (0.014) \\ 
  Education (College) & 0.041^{***} & 0.121^{***} \\ 
  & (0.004) & (0.009) \\ 
  log(Age) & 0.012^{***} & 0.082^{***} \\ 
  & (0.003) & (0.008) \\ 
  Race (Black) & -0.011^{***} & -0.098^{***} \\ 
  & (0.003) & (0.007) \\ 
  Church Attendance & 0.005 & -0.001 \\ 
  & (0.003) & (0.008) \\ 
  Survey Mode (Online) & -0.023^{***} & 0.102^{***} \\ 
  & (0.003) & (0.006) \\ 
  Female * Media & 0.015 & -0.053^{*} \\ 
  & (0.011) & (0.026) \\ 
  Female * Discussions & 0.001 & -0.009 \\ 
  & (0.008) & (0.020) \\ 
  Female * Education & 0.009 & -0.018 \\ 
  & (0.005) & (0.012) \\ 
  Constant & 0.401^{***} & 0.105^{***} \\ 
  & (0.013) & (0.030) \\ 
 \hline \\[-1.8ex] 
Observations & \multicolumn{1}{c}{4,805} & \multicolumn{1}{c}{4,805} \\ 
R$^{2}$ & \multicolumn{1}{c}{0.150} & \multicolumn{1}{c}{0.267} \\ 
\hline 
\hline \\[-1.8ex] 
\textit{Note:}  & \multicolumn{2}{r}{$^{*}$p$<$0.05; $^{**}$p$<$0.01; $^{***}$p$<$0.001} \\ 
\end{tabular} 
\end{table} 


\end{document}